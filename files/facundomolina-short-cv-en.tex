%%%%%%%%%%%%%%%%%%%%%%%%%%%%%%%%%%%%%%%%%
% Plasmati Graduate CV
% LaTeX Template
% Version 1.0 (24/3/13)
%
% This template has been downloaded from:
% http://www.LaTeXTemplates.com
%
% Original author:
% Alessandro Plasmati (alessandro.plasmati@gmail.com)
%
% License:
% CC BY-NC-SA 3.0 (http://creativecommons.org/licenses/by-nc-sa/3.0/)
%
% Important note:
% This template needs to be compiled with XeLaTeX.
% The main document font is called Fontin and can be downloaded for free
% from here: http://www.exljbris.com/fontin.html
%
%%%%%%%%%%%%%%%%%%%%%%%%%%%%%%%%%%%%%%%%%

%----------------------------------------------------------------------------------------
%	PACKAGES AND OTHER DOCUMENT CONFIGURATIONS
%----------------------------------------------------------------------------------------

\documentclass[a4paper,10pt]{article} % Default font size and paper size

\usepackage[utf8]{inputenc}

\usepackage{amsfonts}
\usepackage{longtable}
\usepackage[usenames,dvipsnames]{xcolor} % Required for specifying custom colors

\usepackage{fullpage}
% To reduce the height of the top margin uncomment: \addtolength{\voffset}{-1.3cm}

\usepackage{hyperref} % Required for adding links	and customizing them
\definecolor{linkcolour}{rgb}{0,0.2,0.6} % Link color
\hypersetup{colorlinks,breaklinks,urlcolor=linkcolour,linkcolor=linkcolour} % Set link colors throughout the document

\usepackage{titlesec} % Used to customize the \section command
\titleformat{\section}{\Large\scshape\raggedright}{}{0em}{}[\titlerule] % Text formatting of sections
\titleformat{\subsection}{\large\scshape\raggedright}{}{1em}{\underline}[] % Text formatting of subsections
\titlespacing{\section}{0pt}{3pt}{5pt} % Spacing around sections
\titlespacing{\subsection}{0pt}{10pt}{2pt} % Spacing around subsections
\usepackage{graphicx}
\begin{document}

\pagestyle{empty} % Removes page numbering

%----------------------------------------------------------------------------------------
%	NAME AND CONTACT INFORMATION
%----------------------------------------------------------------------------------------

{\raggedleft{\Huge Facundo \textsc{Molina}}} \\
\textit{Postdoctoral Researcher} \\
\textsc{Email:} \href{mailto:facundo.molina@imdea.org}{facundo.molina@imdea.org} \\
\textsc{Website:} \href{https://facumolina.github.io}{https://facumolina.github.io}

\section{Employment History}

\begin{tabular}{rl}
\\
\textsc{Since 2022}	& \textbf{Postdoctoral Researcher} \\
& \textit{IMDEA Software Institute}, Madrid, Spain. \\

\textsc{2018 - 2022} & \textbf{Teaching Assistant} \\ & \textit{Department of Computer Science, University of Río Cuarto}, Argentina. \\

\textsc{2014 - 2017} & \textbf{Student Teaching Assistant} \\
& \textit{Department of Computer Science, University of Río Cuarto}, Argentina. \\
\end{tabular}

%----------------------------------------------------------------------------------------
%	EDUCATION
%----------------------------------------------------------------------------------------

\section{Education}

\begin{tabular}{rl}
\\
\textsc{2017 - 2022}	& \textbf{Ph.D. in Computer Science} \\
& \textit{Faculty of Mathematics, Astronomy, Physics and Computing, University of Córdoba}, Argentina. \\

\textsc{2012 - 2017}	& \textbf{Undergraduate degree in Computer Science} (5-year + thesis) \\
& \textit{Department of Computer Science, University of Río Cuarto}, Argentina. GPA: 9.43. \\

\textsc{2012 - 2014}	& \textbf{Undergraduate degree in Computer Science} (3-year + final project) \\
& \textit{Department of Computer Science, University of Río Cuarto}, Argentina. GPA: 9.33. \\

\end{tabular}

\section{Publications (5 more relevant)}
\begin{longtable}{rl}

\textsc{November} 2022  & \textbf{Learning to Prune Infeasible Paths in Generalized Symbolic Execution} \\
	& F. Molina, P. Ponzio, N. Aguirre and M. Frias. \textit{{IEEE} 33rd International Symposium} \\
	& \textit{on Software Reliability, {ISSRE} 2022, Charlotte, USA.} \\

\textsc{May} 2022  & \textbf{Fuzzing Class Specifications}. F. Molina, M. d'Amorim and N. Aguirre. \\
	& \textit{ACM/IEEE International Conference on Software Engineering, ICSE 2022, Pittsburgh, USA.} \\

\textsc{May} 2021  & \textbf{EvoSpex: An Evolutionary Algorithm for Learning Postconditions} \\
& F. Molina, P. Ponzio, N. Aguirre and M. Frias. \textit{ACM/IEEE International Conference} \\
& \textit{on Software Engineering, ICSE 2021, Madrid, Spain.} \\

\textsc{July} 2019  & \textbf{An Evolutionary Approach to Translating Operational Specifications} \\ 
& \textbf{into Declarative Specifications}. F. Molina, C. Cornejo, R. Degiovanni, \\ 
& G. Regis, P. Castro, N. Aguirre and M. Frias. \textit{Science of Computer Programming 2019.} \\

\textsc{May} 2019  & \textbf{Training Binary Classifiers as Data Structure Invariants} \\ 
& F. Molina, P. Ponzio, R. Degiovanni, G. Regis, N. Aguirre and M. Frias \\ 
& \textit{International Conference on Software Engineering, ICSE 2019, Montreal, Canada.} \\

\end{longtable}

%\section{Recent Funding}
%\begin{longtable}{rl}

%\textsc{2022} & \textbf{Modular Bounded Verification with Expressive Contracts} \\
%& \textit{Amazon Research Awards ARA 2022. Amazon program providing unrestricted funds} \\ 
%& \textit{and AWS Promotional Credits to academic researchers.} \\
%& Principal Investigator: Marcelo Frias. \\

%\end{longtable}

\section{Awards \& Distinctions}
\begin{longtable}{rl}

\textsc{2020} & \textbf{Latin America PhD Award} \\
& \textit{A research award for PhD students at universities in Latin America, granted by Microsoft Research.} \\

\end{longtable}

\section{Services}
\begin{longtable}{rl}
\textsc{2021}   & \textbf{Program Committee} \\
& \textit{1st International Workshop on Test Oracles, TORACLE 2021 (at ESEC/FSE 2021).} \\

\textsc{2017 - 2022} & \textbf{Student Volunteer} \\
& \textit{International Conference on Software Engineering - ICSE 2017, ICSE 2019, ICSE 2021.} \\
& \textit{International Conference on Automated Software Engineering - ASE 2018.} \\

\end{longtable}

\section{Software}
\begin{longtable}{rl}
\textsc{ISSRE 2022} & \textbf{\url{https://sites.google.com/view/learning-symbolic-invariants}} \\
& \textit{Learning approach to learn to prune infeasible paths} \\
& \textit{in generalized symbolic execution, presented at ISSRE 2022.} \\

\textsc{ICSE 2022} & \textbf{SpecFuzzer: \url{https://sites.google.com/view/specfuzzer}} \\
& \textit{Fuzzing based technique for inferring class specifications, presented at ICSE 2022.} \\

\textsc{ICSE 2021} & \textbf{EvoSpex: \url{https://sites.google.com/view/evospex}} \\
& \textit{Evolutionary algorithm for learning postconditions of Java methods, presented at ICSE 2021.} \\

\textsc{ICSE 2019} & \textbf{\url{https://sites.google.com/site/learninginvariants}} \\
& \textit{Technique to train learning models to capture data structure invariants, presented at ICSE 2019.} \\

\end{longtable}

\end{document}
