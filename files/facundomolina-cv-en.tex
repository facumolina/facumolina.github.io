%%%%%%%%%%%%%%%%%%%%%%%%%%%%%%%%%%%%%%%%%
% Plasmati Graduate CV
% LaTeX Template
% Version 1.0 (24/3/13)
%
% This template has been downloaded from:
% http://www.LaTeXTemplates.com
%
% Original author:
% Alessandro Plasmati (alessandro.plasmati@gmail.com)
%
% License:
% CC BY-NC-SA 3.0 (http://creativecommons.org/licenses/by-nc-sa/3.0/)
%
% Important note:
% This template needs to be compiled with XeLaTeX.
% The main document font is called Fontin and can be downloaded for free
% from here: http://www.exljbris.com/fontin.html
%
%%%%%%%%%%%%%%%%%%%%%%%%%%%%%%%%%%%%%%%%%

%----------------------------------------------------------------------------------------
%	PACKAGES AND OTHER DOCUMENT CONFIGURATIONS
%----------------------------------------------------------------------------------------

\documentclass[a4paper,10pt]{article} % Default font size and paper size

\usepackage[utf8]{inputenc}

\usepackage{amsfonts}
\usepackage{longtable}
\usepackage[usenames,dvipsnames]{xcolor} % Required for specifying custom colors

\usepackage{fullpage}
% To reduce the height of the top margin uncomment: \addtolength{\voffset}{-1.3cm}

\usepackage{hyperref} % Required for adding links	and customizing them
\definecolor{linkcolour}{rgb}{0,0.2,0.6} % Link color
\hypersetup{colorlinks,breaklinks,urlcolor=linkcolour,linkcolor=linkcolour} % Set link colors throughout the document

\usepackage{titlesec} % Used to customize the \section command
\titleformat{\section}{\Large\scshape\raggedright}{}{0em}{}[\titlerule] % Text formatting of sections
\titleformat{\subsection}{\large\scshape\raggedright}{}{1em}{\underline}[] % Text formatting of subsections
\titlespacing{\section}{0pt}{3pt}{5pt} % Spacing around sections
\titlespacing{\subsection}{0pt}{10pt}{2pt} % Spacing around subsections
\usepackage{graphicx}
\begin{document}

\pagestyle{empty} % Removes page numbering

%----------------------------------------------------------------------------------------
%	NAME AND CONTACT INFORMATION
%----------------------------------------------------------------------------------------

{\raggedleft{\Huge Facundo \textsc{Molina}}} \\
\textit{Postdoctoral Researcher} \\
\textsc{Email:} \href{mailto:facundo.molina@imdea.org}{facundo.molina@imdea.org} \\
\textsc{Website:} \href{https://facumolina.github.io}{https://facumolina.github.io} \\

{\raggedleft
\textsc{Research Statement:} my main research interests are in the area of Software Testing and Analysis and the use of Artificial Intelligence for Software Engineering (AI4SE), with the goal of improving software reliability and quality.} \\

\section{Employment History}

\begin{longtable}{ll}
\\
2022 - & \textbf{Postdoctoral Researcher} - \textit{IMDEA Software Institute}, Madrid, Spain. \\
& I do research on software testing and analysis, which includes developing and implementing \\ 
& novel program analysis techniques, and disseminating our research on top tier software \\ 
& engineering  conferences and journals. Some projects I participate are: \\ 
& \textbf{-} \textit{FixCheck:} patch correctness assessment improvement based on random testing and Large \\
& Language Models (LLMs). A \href{https://github.com/facumolina/fixcheck}{tool} is implemented using Java, Python, Docker and also using \\
& \href{https://huggingface.co/}{Hugging Face} and \href{https://github.com/ggerganov/llama.cpp}{llama.cpp} for supporting LLMs. \\
& \textbf{-} \textit{MemoRIA:} automated inference of metamorphic oracles combining grammar-based fuzzing, \\
& dynamic analysis and SAT-based analysis. A \href{https://zenodo.org/records/10683011}{prototype} is implemented using Java and Python, \\ 
& and using \href{https://randoop.github.io/randoop/}{Randoop} for the dynamic analysis and \href{https://alloytools.org/}{Alloy} for the SAT-based analysis. \\
& \textbf{-} \textit{PLI:} symbolic execution for programs that manipulate complex heap-allocated data. A \\ 
& \href{https://github.com/JuanmaCopia/spf-pli}{prototype} is implemented in Java on top of the \href{https://github.com/SymbolicPathFinder/jpf-symbc}{Symbolic PathFinder} engine. \\ & \\

2017 - & \textbf{Ph.D. Student} - \textit{FaMaF, University of Córdoba}, Argentina. \\ 
2022 & My PhD was focused on developing techniques for the \textit{automated generation of test oracles} using \\
& search-based and learning-based techniques (evolutionary algorithms and neural nets). My \\
& advisor was Prof. Nazareno Aguirre and my dissertation is available in the UNC \href{https://rdu.unc.edu.ar/handle/11086/26692}{digital repository}. \\ 
& The main contributions of my thesis are: \\ 
& \textbf{-} \textit{SpecFuzzer:} automated inference of test oracles in the form of class specifications using \\
& grammar-based fuzzing. A \href{https://github.com/facumolina/specfuzzer}{tool} is implemented using Java and Python, and built on top of the \\ 
& \href{https://plse.cs.washington.edu/daikon/}{Daikon} dynamic invariant detector. \\ 
& \textbf{-} \textit{EvoSpex:} search-based inference of postconditions of Java methods using genetic algorithms. \\
& A \href{https://github.com/facumolina/evospex}{tool} is implemented in Java using the \href{https://homepages.ecs.vuw.ac.nz/~lensenandr/jgap/documentation/}{JGAP} library. \\ 
& \textbf{-} \textit{NNInvs:} data structure object classification using artificial neural networks. One \href{https://sites.google.com/site/learninginvariants}{prototype} is \\
& implemented using Java and Python, relying on the \href{https://scikit-learn.org/stable/index.html}{scikit-learn} library. A second \href{https://sites.google.com/site/learninginvariants}{prototype} is \\ 
& developed (to be used in the context of symbolic execution) using the \href{https://keras.io/}{Keras} deep learning library. \\ & \\

2014 - & \textbf{Teaching Assistant} - \textit{Department of Computer Science, University of Río Cuarto}, Argentina. \\ 
2022 & I was a Teaching assistant different courses of a Computer Science degree, including Computability \\
& and Complexity, Distributed and Outsourced Software Engineering, and Introduction to \\
& Programming. My teaching duties involved being in charge of practical classes where students had \\  
& to solve assignments. Before graduating in 2017, I was a Student teaching assistant in courses such \\ 
& as Data Structures and Algorithms, Algorithms Design Techniques, Programming Paradigms and \\
& System Design and Analysis where I helped other teacher assistants. \\ & \\

2015 - & \textbf{Software Engineer} - \textit{SMF Consulting S.L}. \\ 
2019 & Java Software Developer providing solutions to different customers based on an ERP platform. \\ 
& During this time I worked with different teams implementing backend solutions in Java (CRUD \\ 
& operations, API definitions, and integrations with other platforms), maintaining a PostgreSQL \\
& database, extending and improving a frontend for retail operations using JavaScript, and  \\
& performing DevOps taks managing AWS and Vultr servers. \\ & \\

%2014 - & \textbf{Student Teaching Assistant} - \textit{Department of Computer Science, University of} \\ 
%2017 & \textit{Río Cuarto}, Argentina. Student teaching assistant in different courses of a degree in \\ 
%& Computer Science including Algorithms I, Algorithms II, Comparative Analysis of \\ 
%& Languages and Analysis and Design of Systems. 
%\\ & \\
\end{longtable}

%----------------------------------------------------------------------------------------
%	EDUCATION
%----------------------------------------------------------------------------------------

\section{Education}

\begin{tabular}{ll}
\\
2017 - & \textbf{Ph.D., Computer Science} \\
2022 & Dissertation: Techniques based on Learning and Search for Specification Inference. \\
& \textit{Faculty of Mathematics, Astronomy, Physics and Computing - FaMaF} \\ & \textit{University of Córdoba} - Argentina \\ & \\

2012 - & \textbf{Computer Science Licenciate} \\
2017 & (5-year + thesis undergraduate program of study) \\
& \textit{Department of Computer Science - FCEFQyN} \\ & \textit{University of Río Cuarto} - Argentina \\ 
& \textsc{Thesis project:} Automatic Learning of Relational Specifications using \\ 
& Evolutionary Computation. \\ 
& \textsc{Average score:} 9.43 out of 10 \\ & \\

2012 -	& \textbf{B.S. in Computer Science} \\
2014 & \textit{Department of Computer Science - FCEFQyN} \\ 
& \textit{University of Río Cuarto} - Argentina \\
& \textsc{Thesis project:} Project in the course of Distributed and Outsourced \\ 
& Software Engineering. \\ 
& \textsc{Average score:} 9.33 out of 10 \\ \\ 

\end{tabular}

\section{Publications}
\begin{longtable}{rl}

\textsc{July} 2024  & \textbf{Abstraction-Aware Inference of Metamorphic Relations} \\
        & Agustín Nolasco, Facundo Molina, Renzo Degiovanni, Alessandra Gorla, \\ 
        & Diego Garbervetsky, Mike Papadakis, Sebastian Uchitel, Nazareno Aguirre \\
        & and Marcelo F. Frias. \\
        & \textit{ACM Conference on the Foundations of Software Engineering, FSE 2024,} \\
        & \textit{Porto de Galinhas, Brazil, July 15 - 19, 2024.} \href{https://dl.acm.org/doi/10.1145/3643747}{[doi]} \\ & \\

\textsc{May} 2024  & \textbf{Improving Patch Correctness Analysis via Random Testing} \\
        & \textbf{and Large Language Models} \\
        & Facundo Molina, Juan Manuel Copia and Alessandra Gorla. \\
        & \textit{To appear in the 17th IEEE International Conference on Software Testing,} \\
        & \textit{Verification and Validation, ICST 2024, Toronto, Canada, May 27 - 31, 2024.} \href{https://facumolina.github.io/files/MOLINA_ETAL_ICST2024.pdf}{[pdf]} \\ & \\

\textsc{October} 2023  & \textbf{Enabling Efficient Assertion Inference} \\
        & Aayush Garg, Renzo Degiovanni, Facundo Molina, Maxime Cordy,\\
        & Nazareno Aguirre, Mike Papadakis, and Yves Le Traon. \\
        & \textit{IEEE 34th International Symposium on Software Reliability Engineering,} \\
        & \textit{ISSRE 2023, Florence, Italy, October 9 - 12, 2023.} \href{https://doi.ieeecomputersociety.org/10.1109/ISSRE59848.2023.00039}{[doi]}\\ & \\

\textsc{October} 2023  & \textbf{Precise Lazy Initialization for Programs with Complex Heap Inputs} \\
        & Juan Manuel Copia, Facundo Molina, Nazareno Aguirre, Marcelo F. Frias,\\
        & Alessandra Gorla, and Pablo Ponzio. \\
        & \textit{IEEE 34th International Symposium on Software Reliability Engineering,} \\
        & \textit{ISSRE 2023, Florence, Italy, October 9 - 12, 2023.} \href{https://doi.ieeecomputersociety.org/10.1109/ISSRE59848.2023.00080}{[doi]}\\ & \\

\textsc{September} 2023  & \textbf{SpecFuzzer: A Tool for Inferring Class Specifications via Grammar} \\
        & \textbf{Based Fuzzing} - Facundo Molina, Nazareno Aguirre and Marcelo d'Amorim \\
        & \textit{IEEE/ACM 38th International Conference on Automated Software Engineering,} \\
        & \textit{ASE 2023, Luxembourg, September 11 - 15, 2023.} \href{https://doi.ieeecomputersociety.org/10.1109/ASE56229.2023.00024}{[doi]}\\ & \\

\textsc{July} 2023  & \textbf{EvoSpex: A Search-based Tool for Postcondition Inference} \\
& Facundo Molina, Pablo Ponzio, Nazareno Aguirre and Marcelo F. Frias. \\
& \textit{ACM SIGSOFT 32nd International Symposium on Software Testing and Analysis,} \\
& \textit{ISSTA 2023, Seattle, USA, July 17 - 21, 2023.} \href{https://dl.acm.org/doi/abs/10.1145/3597926.3604928}{[doi]} \\ & \\

\textsc{April} 2023  & \textbf{Efficient Bounded Exhaustive Input Generation from Program APIs} \\
        & Mariano Politano, Valeria Bengolea, Facundo Molina, Marcelo F. Frias,\\
        & Nazareno Aguirre, and Pablo Ponzio. \\
	& \textit{26th International Conference on Fundamental Approaches to Software Engineering,} \\
        & \textit{FASE 2023, Paris, France, April 22 - 27, 2023.} \href{https://doi.org/10.1007/978-3-031-30826-0_6}{[doi]}\\ & \\

\textsc{November} 2022  & \textbf{Learning to Prune Infeasible Paths in Generalized Symbolic Execution} \\
        & Facundo Molina, Pablo Ponzio, Nazareno Aguirre and Marcelo F. Frias.\\
        & \textit{IEEE 33rd International Symposium on Software Reliability Engineering,} \\
        & \textit{ISSRE 2022, Charlotte, NC, USA, October 31 - Nov. 3, 2022.} \href{https://doi.org/10.1109/ISSRE55969.2022.00054}{[doi]} \\ & \\

\textsc{May} 2022  & \textbf{Fuzzing Class Specifications} \\
	& Facundo Molina, Marcelo d'Amorim and Nazareno Aguirre. \\
	& \textit{Proceedings of the 44th ACM/IEEE International Conference on Software} \\
	& \textit{Engineering, ICSE 2022, Pittsburgh, USA, May 22-27, 2022.} \href{https://dl.acm.org/doi/abs/10.1145/3510003.3510120}{[doi]}\\ & \\

\textsc{May} 2021  & \textbf{EvoSpex: An Evolutionary Algorithm for Learning Postconditions} \\
& Facundo Molina, Pablo Ponzio, Nazareno Aguirre and Marcelo F. Frias. \\
& \textit{Proceedings of the 43rd ACM/IEEE International Conference on Software} \\
& \textit{Engineering, ICSE 2021, Madrid, Spain, May 23-29, 2021.} \href{https://doi.ieeecomputersociety.org/10.1109/ICSE43902.2021.00112}{[doi]} \\ & \\

\textsc{September} 2020  & \textbf{Applying Learning Techniques to Oracle Synthesis} \\
& Facundo Molina \\
& \textit{Doctoral Symposium, Proceedings of the 35th IEEE/ACM International Conference} \\
& \textit{on Automated Software Engineering, ASE 2020, Australia, September 21-25, 2020.} \href{https://dl.acm.org/doi/10.1145/3324884.3415287}{[doi]} \\ & \\

\textsc{July} 2019  & \textbf{An Evolutionary Approach to Translating Operational Specifications} \\ & \textbf{into Declarative Specifications} - Facundo Molina, César Cornejo, Renzo\\ 
& Degiovanni, Germán Regis, Pablo Castro, Nazareno Aguirre and Marcelo Frias \\
& \textit{Science of Computer Programming, Volume 181, Pages 47-63, 2019.} \href{https://www.sciencedirect.com/science/article/pii/S0167642319300735}{[doi]} \\ & \\

\textsc{May} 2019  & \textbf{Training Binary Classifiers as Data Structure Invariants} \\ 
& Facundo Molina, Pablo Ponzio, Renzo Degiovanni, Germán Regis, \\ 
& Nazareno Aguirre and Marcelo Frias \\
& \textit{Proceedings of the 41th International Conference on Software Engineering,} \\
& \textit{ICSE 2019, Montreal, Canada, May 25-31, 2019.} \href{https://doi.ieeecomputersociety.org/10.1109/ICSE.2019.00084}{[doi]} \\ & \\

\textsc{September} 2018  & \textbf{A Genetic Algorithm for Goal-Conflict Identification} \\ 
& Renzo Degiovanni, Facundo Molina, Germán Regis and Nazareno Aguirre \\
& \textit{Proceedings of the 33rd ACM/IEEE International Conference on Automated } \\
& \textit{Software Engineering, ASE 2018, Montpellier, France, September 3-7, 2018.} \href{https://doi.org/10.1145/3238147.3238220}{[doi]} \\ & \\

\textsc{May} 2018  & \textbf{From Operational to Declarative Specifications using a Genetic} \\ & \textbf{Algorithm} - Facundo Molina, Renzo Degiovanni, Germán Regis, Pablo Castro,\\
& Nazareno Aguirre and Marcelo Frias \\
& \textit{Proceedings of the 11th International Workshop on Search-Based Software Testing,} \\
& \textit{SBST@ICSE 2018, Gothenburg, Sweden, May 28-29, 2018.} \href{https://dl.acm.org/doi/10.1145/3194718.3194725}{[doi]} \\ & \\

\textsc{November} 2016 & \textbf{An Evolutionary Approach to Translate Operational Specifications} \\ & \textbf{into Declarative Specifications} - Facundo Molina, César Cornejo, Renzo \\
& Degiovanni, Germán Regis, Pablo Castro, Nazareno Aguirre and Marcelo Frias \\
& \textit{Proceedings of the 19th Brazilian Symposium on Formal Methods} \\ 
& \textit{SBMF 2016, Natal, Brazil, November 22-25, 2016.} \href{https://doi.org/10.1007/978-3-319-49815-7_9}{[doi]} \\ & \\

\end{longtable}

\section{Public Talks}

\begin{longtable}{rl}

\textsc{July} 2024  & \textbf{Abstraction-aware Inference of Metamorphic Relations} \\ 
& Research track, FSE conference, Porto de Galinhas, Brazil. \\ & \\

\textsc{May} 2024  & \textbf{Improving Patch Correctness Analysis via Random Testing and} \\ 
& \textbf{Large Language Models} - Research track, ICST conference, Toronto, Canada. \\ & \\

\textsc{October} 2023  & \textbf{Automated Generation of Test Oracles} \\ 
& Invited speaker at the \textit{Jornadas de Ciencias de la Computación}, JCC 2023, \\ 
& Rosario, Argentina. \\ & \\

\textsc{September} 2023  & \textbf{SpecFuzzer: A Tool for Inferring Class Specifications via} \\ 
& \textbf{Grammar-based Fuzzing} - Tool Demonstrations track, ASE conference, Luxembourg. \\ & \\

\textsc{July} 2023 & \textbf{EvoSpex: A Search-based Tool for Postcondition Inference} \\ 
& Tool Demonstrations track, ISSTA conference, Seattle, USA. \\ & \\

\textsc{November} 2022 & \textbf{Learning to Prune Infeasible Paths in Generalized Symbolic Execution} \\
& Research track, ISSRE conference, Charlotte, USA. \\ & \\

\textsc{October} 2022 & \textbf{Fuzzing Class Specifications} - Oral communication, \\ 
& Simposio Argentino de Ingeniería de Software ASSE 2022 (virtual), Argentina. \\ & \\

\textsc{May} 2022 & \textbf{Fuzzing Class Specifications} - Research track, ICSE conference, Pittsburgh, USA. \\ & \\

\textsc{March} 2022 & \textbf{EvoSpex: An Evolutionary Algorithm for Learning Postconditions} \\ 
& Invited talk, Argentine Workshop on Fundamentals for the Automatic Analysis \\ 
& and Construction of Software FACAS 2022, La Falda, Argentina. \\ & \\

\textsc{October} 2021 & \textbf{EvoSpex: An Evolutionary Algorithm for Learning Postconditions} \\ 
& Oral communication, Simposio Argentino de Ingeniería de Software ASSE 2021 (virtual), \\ 
& Argentina. \\ & \\

\textsc{May} 2021 & \textbf{EvoSpex: An Evolutionary Algorithm for Learning Postconditions} \\ 
& Research track, ICSE conference (virtual), Madrid, Spain. \\ & \\

\textsc{September} 2020 & \textbf{Applying Learning Techniques to Oracle Synthesis} \\ 
& Doctoral symposium, ASE conference (virtual), Melbourne, Australia. \\ & \\

\textsc{May} 2019 & \textbf{Training Binary Classifiers as Data Structure Invariants} \\ 
& Research track, ICSE conference, Montréal, Canada. \\ & \\

\textsc{March} 2019 & \textbf{Learning Hybrid Invariants to Improve Symbolic Execution on Structurally} \\ 
& \textbf{Complex Inputs} - Invited talk, Argentine Workshop on Fundamentals for the Automatic \\ 
& Analysis and Construction of Software FACAS 2021, La Falda, Argentina. \\ & \\

\textsc{September} 2018 & \textbf{A Genetic Algorithm for Goal-Conflict Identification} \\ 
& Research track, ASE conference, Montpellier, France. \\ & \\

\textsc{November} 2016 & \textbf{An Evolutionary Approach to Translate Operational Specifications into} \\
& \textbf{Declarative Specifications} - Research track, Brazilian Symposium on Formal \\ 
& Methods SBMF 2016, Natal, Brazil. \\ & \\

\end{longtable}

\section{Research Prototypes}
\begin{longtable}{rl}

\textbf{FixCheck} & FixCheck is a tool for improving patch correctness analyses in Java. It combines static \\ 
& analysis, random testing and LLMs to
automatically generate tests that highlight \\
& and explain the potential incorrectness of a patch. \\ 
& FixCheck is available at: \href{https://github.com/facumolina/fixcheck}{https://github.com/facumolina/fixcheck} \\ & \\

\textbf{SpecFuzzer} & SpecFuzzer is a tool that automatically infers test oracles in the form of class specifica-\\ 
& tions (postconditions, invariants), and works for Java classes. SpecFuzzer uses a fuzzer \\
& as a generator of candidate assertions; a dynamic invariant detector –Daikon– to filter out \\
& assertions invalidated by a test suite; and a mutation-based mechanism to cluster and \\
& rank assertions, so that similar constraints are grouped and then the stronger prioritized. \\ 
& SpecFuzzer is available at: \href{https://github.com/facumolina/specfuzzer}{https://github.com/facumolina/specfuzzer} \\ & \\


\textbf{EvoSpex} & EvoSpex is a tool that, given a Java method, uses an evolutionary algorithm to produce \\ 
& a specification of the method's current behavior, in the form of postcondition assertions. \\
& EvoSpex implements a classic genetic algorithm that searches for a succinct postcondition that \\ 
& accepts the current method behavior, while rejecting any deviation from such behavior. \\ 
& EvoSpex is available at: \href{https://github.com/facumolina/evospex}{https://github.com/facumolina/evospex} \\ & \\

\textbf{PLI} & PLI is an efficient symbolic execution approach for programs that manipulate complex \\
& heap-allocated data structures with rich structural constraints. PLI works for Java, \\
& and allows preconditions to be specified as standard operational predicates for concrete \\
& structures, eliminating the need for additional specifications tailored to symbolic heaps. \\
& PLI is available at: \href{https://github.com/JuanmaCopia/spf-pli}{https://github.com/JuanmaCopia/spf-pli} \\ & \\

\end{longtable}

\section{Research Grants \& Scholarships}
\begin{longtable}{rl}

2017 & \textbf{Doctoral Scholarship} \\ 
& \textit{5-year Scholarship granted by Argentina's National Scientific and Technical Research} \\
& \textit{Council (CONICET) to fund doctoral students.} \\ & \\

2016 & \textbf{EVC-CIN Scholarship} \\ 
& \textit{1-year Scolarship granted by the argentinian National Inter University Council (CIN)} \\
& \textit{to encourage undergraduate students to pursue scientific vocations.} \\ & \\

\end{longtable}

\section{Participation in Funded Research Projects}
\begin{longtable}{rl}

11/2023-02/2024 & \textbf{ANZEN: Model-based Safety Analysis through Formal Verification.} \\ 
& This project is a collaboration between IMDEA and Anzen Aerospace \\
& Engineering, SL to explore the use of formal verification tools in the context \\
& of model-based safety analysis. I participate as part of the team from IMDEA Software. \\ & \\

09/2023-08/2027 & \textbf{ESPADA: Efficient and Secure Data Protection Against Digital Attack.} \\ 
& Project lead by Juan Caballero and Alessandra Gorla, granted by the spanish \\ 
& Ministerio de Ciencia e Innovación, co-funded by European Union ESF, EIE \\
& and NextGeneration funds. I participate as a member of the research team. \\ & \\

12/2022-11/2024 & \textbf{PRODIGY: Asegurando la seguridad, escalabilidad y funcionalidad} \\
& \textbf{de los sistemas digitales de procedencia.} \\ 
& Project lead by Juan Caballero y Pedro Moreno-Sánchez, granted by the spanish \\
& Ministerio de Ciencia e Innovación, and co-funded by European Union ESF, EIE \\
& and NextGeneration funds. I participate as a member of the research team. \\ & \\

\end{longtable}

\section{Honors \& Awards}
\begin{longtable}{rl}

\textsc{2020} & \textbf{Latin America PhD Award} \\
& \textit{A research award for PhD students in computing related fields in their 3rd year} \\
& \textit{or beyond at universities in Latin America, and granted by Microsoft Research.} \\ & \\

\textsc{2018} & \textbf{Best Paper Award} \\ 
& \textit{From Operational to Declarative Specifications using a Genetic Algorithm} \\
& \textit{11th International Workshop on Search-Based Software Testing, SBST 2018.} \\ & \\

\textsc{2016} & \textbf{Best Paper Award} \\ 
& \textit{An Evolutionary Approach to Translate Operational Specifications into } \\
& \textit{Declarative Specifications, 19th Brazilian Symposium on Formal Methods, SBMF 2016.} \\ & \\

\textsc{2016} & \textbf{University of Rio Cuarto flag bearer for a 1-year period} \\ 
& \textit{Traditional honour in educational institutions in Argentina to the three top students} \\ 
& \textit{in the institution.} \\ & \\
\end{longtable}

%\section{Teaching Background}
%\begin{longtable}{rl}

%\textsc{March 2018} & \textbf{Teaching Assistant} \\
%\textsc{- June 2022} & \textsc{Courses:} Computability and Complexity, Distributed and Outsourced \\
%& Software Engineering, Introduction to Programming. \\
%& \textit{Department of Computer Science - FCEFQyN} \\
%& \textit{University of Río Cuarto} - Argentina \\ & \\

%\textsc{August 2014} & \textbf{Student Teaching Assistant} \\
%\textsc{- June 2017} & \textsc{Courses:} Data Structures and Algorithms, Algorithms Design Techniques, \\
%& Programming Paradigms, and System Design and Analysis. \\
%& \textit{Department of Computer Science - FCEFQyN} \\
%& \textit{University of Río Cuarto} - Argentina \\ & \\

%\end{longtable}

\section{Supervised Students}
\begin{longtable}{rl}

2024 & \textbf{Claudio Dosantos} - \textit{Undergraduate student} - University of Río Cuarto, Argentina. \\
& Claudio's thesis aims to analyze the effectiveness of regression testing when using \\ 
& different kind of oracles, such as unit assertions and contracts. \\ & \\

2024 & \textbf{Ignacio Gonzalez} - \textit{Undergraduate student} - University of Río Cuarto, Argentina. \\
& Automated test generation tools play a crucial role on dynamic specification inference \\ 
& techniques. Ignacio's work aims at studying how different test generation approaches \\ 
& affects the effectiveness of specification inference techniques. \\ & \\

2023 & \textbf{Agustin Nolasco} - \textit{Undergraduate student} - University of Río Cuarto, Argentina. \\ 
& Agustin's thesis presents a new technique for the inference of metamorphic oracles, \\ 
& based on runtime analysis, grammar-based fuzzing and SAT solving. \\ & \\

\end{longtable}

\section{Academic Service}
\begin{longtable}{rl}

\textsc{2024} & Program committee at \textit{International Conference on AI Foundation Models and Software} \\ 
& \textit{Engineering (FORGE 2024)}. Reviewer at \textit{IEEE Transactions on Software Engineering (TSE)}. \\
& Programm committee of the Industry track at \textit{International Conference on Software} \\
& \textit{Maintenance and Evolution (ICSME 2024).} Artifact Evaluation committee at \textit{International,} \\
& \textit{Conference on Software Engineering (ICSE 2024), International Symposium on Software Testing} \\ 
& \textit{and Analysis (ISSTA 2024).} \\ & \\

\textsc{2023} & Program committee at \textit{International Working Conference on Source Code Analysis and} \\ 
& \textit{Manipulation (SCAM 2023)}. Reviewer at \textit{IEEE Transactions on Software Engineering (TSE)}. \\
& Artifact Evaluation committee at \textit{International Symposium on Software Testing and Analysis,} \\ 
& \textit{(ISSTA 2023), Static Analysis Symposium (SAS 2023).} \\ & \\

\textsc{2022} & Student volunteer at \textit{International Conference on Software Engineering (ICSE 2022.)} \\ & \\ 

\textsc{2021} & Program committee at \textit{International Workshop on Test Oracles (TORACLE 2021)}. \\
& Student volunteer at \textit{International Conference on Software Engineering (ICSE 2021.)} \\ & \\

\textsc{2019} & Student volunteer at \textit{International Conference on Software Engineering (ICSE 2019.)} \\ & \\ 

\textsc{2018} & Student volunteer at \textit{International Conference on Automated Software Engineering (ASE 2018).} \\ & \\ 

\textsc{2017} & Student volunteer at \textit{International Conference on Software Engineering (ICSE 2017.)} \\ & \\ 

\end{longtable}

\section{Extracurricular Courses Taken}
\begin{longtable}{rl}
\textsc{October} 2019 & \textbf{Neural Networks and Deep Learning} - Adjunct Professor Andrew Ng \\
& \textit{Foundations of Deep Learning}  \\
& \textit{An online non-credit course authorized by deeplearning.ai} \\
& \textit{Coursera} \\ & \\

\textsc{March} 2019 & \textbf{Introduction to Data Science in Python} - Christopher Brooks \\
& \textit{Introduction to data manipulation and cleaning techniques using pandas}  \\
& \textit{An online non-credit course authorized by University of Michigan} \\
& \textit{Coursera} \\ & \\

\textsc{August} 2018 & \textbf{Neural Networks} - Dr. Francisco Tamarit \\
\textsc{November} 2018 & \textit{Mathematical Foundations of Artificial Neural Networks}  \\ 
& \textit{Postgraduate courses} \\
& \textit{University of Córdoba} - Argentina \\ & \\

\textsc{August} 2017 & \textbf{Text Mining} - Dr. Laura Alonso Alemany \\
\textsc{November} 2017 & \textit{Text Mining techniques applied to Natural Language Processing problems (Word}  \\ 
& \textit{similarity, Document clustering, Sense discrimination, Machine translation)} \\
& \textit{Postgraduate courses} \\
& \textit{University of Córdoba} - Argentina \\ & \\

\textsc{August} 2017 & \textbf{Information and its Demons} - Dr. Javier Blanco \\
\textsc{November} 2017 & \textit{Information Philosophy} \\
& \textit{Postgraduate courses} \\
& \textit{University of Río Cuarto} - Argentina \\ & \\

\textsc{February} 2017 & \textbf{Human Dynamics: Data, Networks and Modelling} - Dr. Márton Karsai \\
& \textit{Summer School of Computer Science RIO 2017} \\
& \textit{University of Río Cuarto} - Argentina \\ & \\

\textsc{March} 2016 & \textbf{Software Testing} - Dr. Renzo Degiovanni \\
\textsc{June} 2016 & \textit{Main software testing techniques using state-of-the-art tools} \\
& \textit{Postgraduate courses} \\
& \textit{University of Río Cuarto} - Argentina \\ & \\

\textsc{February} 2016 & \textbf{Systematic Test Case Generation} - Prof. Sarfraz Khurshid \\
& \textit{Summer School of Computer Science RIO 2016} \\
& \textit{University of Río Cuarto} - Argentina \\ & \\

\textsc{February} 2016 & \textbf{Symbolic Program Analysis} - Prof. Willem Visser \\
& \textit{Summer School of Computer Science RIO 2016} \\
& \textit{University of Río Cuarto} - Argentina \\ & \\

\textsc{February} 2015 & \textbf{Description Logic Reasoning} - Dr. Anni-Yasmin Turhan\\
& \textit{Summer School of Computer Science RIO 2015} \\
& \textit{University of Río Cuarto} - Argentina \\ & \\

\textsc{February} 2015 & \textbf{Fundamentals of Quantum Programming Languages} - Dr. Alejandro Díaz-Caro \\
& \textit{Summer School of Computer Science RIO 2015} \\
& \textit{University of Río Cuarto} - Argentina \\ & \\

\end{longtable}

\section{Languages}
\begin{tabular}{rl}
\\
\textsc{Spanish:} & Mother tongue\\
\textsc{English:} & Fluent \\ & \\
\end{tabular}

\end{document}
