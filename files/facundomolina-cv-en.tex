%%%%%%%%%%%%%%%%%%%%%%%%%%%%%%%%%%%%%%%%%
% Plasmati Graduate CV
% LaTeX Template
% Version 1.0 (24/3/13)
%
% This template has been downloaded from:
% http://www.LaTeXTemplates.com
%
% Original author:
% Alessandro Plasmati (alessandro.plasmati@gmail.com)
%
% License:
% CC BY-NC-SA 3.0 (http://creativecommons.org/licenses/by-nc-sa/3.0/)
%
% Important note:
% This template needs to be compiled with XeLaTeX.
% The main document font is called Fontin and can be downloaded for free
% from here: http://www.exljbris.com/fontin.html
%
%%%%%%%%%%%%%%%%%%%%%%%%%%%%%%%%%%%%%%%%%

%----------------------------------------------------------------------------------------
%	PACKAGES AND OTHER DOCUMENT CONFIGURATIONS
%----------------------------------------------------------------------------------------

\documentclass[a4paper,10pt]{article} % Default font size and paper size

\usepackage[utf8]{inputenc}

\usepackage{amsfonts}
\usepackage{longtable}
\usepackage[usenames,dvipsnames]{xcolor} % Required for specifying custom colors

\usepackage{fullpage}
% To reduce the height of the top margin uncomment: \addtolength{\voffset}{-1.3cm}

\usepackage{hyperref} % Required for adding links	and customizing them
\definecolor{linkcolour}{rgb}{0,0.2,0.6} % Link color
\hypersetup{colorlinks,breaklinks,urlcolor=linkcolour,linkcolor=linkcolour} % Set link colors throughout the document

\usepackage{titlesec} % Used to customize the \section command
\titleformat{\section}{\Large\scshape\raggedright}{}{0em}{}[\titlerule] % Text formatting of sections
\titleformat{\subsection}{\large\scshape\raggedright}{}{1em}{\underline}[] % Text formatting of subsections
\titlespacing{\section}{0pt}{3pt}{5pt} % Spacing around sections
\titlespacing{\subsection}{0pt}{10pt}{2pt} % Spacing around subsections
\usepackage{graphicx}
\begin{document}

\pagestyle{empty} % Removes page numbering

%----------------------------------------------------------------------------------------
%	NAME AND CONTACT INFORMATION
%----------------------------------------------------------------------------------------

{\raggedleft{\Huge Facundo \textsc{Molina}}} \\
\textit{Postdoctoral Researcher} \\
\textsc{Email:} \href{mailto:facundo.molina@imdea.org}{facundo.molina@imdea.org} \\
\textsc{Website:} \href{https://facumolina.github.io}{https://facumolina.github.io}


\section{Employment History}

\begin{tabular}{rl}
\\
\textsc{Since 2022}     & \textbf{Postdoctoral Researcher} \\
& Research area: Software testing and analysis. \\
& \textit{IMDEA Software Institute}, Madrid, Spain. \\ & \\

\textsc{2018 - 2022} & \textbf{Teaching Assistant} \\ & \textit{Department of Computer Science, University of Río Cuarto}, Argentina. \\ & \\

\textsc{2014 - 2017} & \textbf{Student Teaching Assistant} \\
& \textit{Department of Computer Science, University of Río Cuarto}, Argentina. \\ & \\
\end{tabular}

%----------------------------------------------------------------------------------------
%	EDUCATION
%----------------------------------------------------------------------------------------

\section{Education}

\begin{tabular}{rl}
\\
\textsc{2017 - 2022}	& \textbf{Ph.D., Computer Science} \\
& Dissertation: Techniques based on Learning and Search for Specification Inference. \\
& \textit{Faculty of Mathematics, Astronomy, Physics and Computing - FaMaF} \\ & \textit{University of Córdoba} - Argentina \\ & \\

\textsc{2012 - 2017}	& \textbf{Computer Science Licenciate} \\
& (5-year + thesis undergraduate program of study) \\
& \textit{Department of Computer Science - FCEFQyN} \\ & \textit{University of Río Cuarto} - Argentina \\ 
& \textsc{Thesis project:} Automatic Learning of Relational Specifications using \\ 
& Evolutionary Computation. \\ 
& \textsc{Average score:} 9.43 out of 10 \\ & \\

\textsc{2012 - 2014}	& \textbf{B.S. in Computer Science} \\
& \textit{Department of Computer Science - FCEFQyN} \\ 
& \textit{University of Río Cuarto} - Argentina \\
& \textsc{Thesis project:} Project in the course of Distributed and Outsourced \\ 
& Software Engineering. \\ 
& \textsc{Average score:} 9.33 out of 10 \\ \\ 

\end{tabular}

\section{Publications}
\begin{longtable}{rl}

\textsc{May} 2024  & \textbf{Improving Patch Correctness Analysis via Random Testing} \\
        & \textbf{and Large Language Models} \\
        & Facundo Molina, Juan Manuel Copia and Alessandra Gorla. \\
        & \textit{To appear in the 17th IEEE International Conference on Software Testing,} \\
        & \textit{Verification and Validation, ICST 2024, Toronto, Canada, May 27 - 31, 2024.} \\ & \\

\textsc{October} 2023  & \textbf{Enabling Efficient Assertion Inference} \\
        & Aayush Garg, Renzo Degiovanni, Facundo Molina, Maxime Cordy,\\
        & Nazareno Aguirre, Mike Papadakis, and Yves Le Traon. \\
        & \textit{IEEE 34th International Symposium on Software Reliability Engineering,} \\
        & \textit{ISSRE 2023, Florence, Italy, October 9 - 12, 2023.} \\ & \\

\textsc{October} 2023  & \textbf{Precise Lazy Initialization for Programs with Complex Heap Inputs} \\
        & Juan Manuel Copia, Facundo Molina, Nazareno Aguirre, Marcelo F. Frias,\\
        & Alessandra Gorla, and Pablo Ponzio. \\
        & \textit{IEEE 34th International Symposium on Software Reliability Engineering,} \\
        & \textit{ISSRE 2023, Florence, Italy, October 9 - 12, 2023.} \\ & \\

\textsc{April} 2023  & \textbf{Efficient Bounded Exhaustive Input Generation from Program APIs} \\
        & Mariano Politano, Valeria Bengolea, Facundo Molina, Marcelo F. Frias,\\
        & Nazareno Aguirre, and Pablo Ponzio. \\
	& \textit{26th International Conference on Fundamental Approaches to Software Engineering,} \\
        & \textit{FASE 2023, Paris, France, April 22 - 27, 2023.} \\ & \\

\textsc{November} 2022  & \textbf{Learning to Prune Infeasible Paths in Generalized Symbolic Execution} \\
        & Facundo Molina, Pablo Ponzio, Nazareno Aguirre and Marcelo F. Frias.\\
        & \textit{IEEE 33rd International Symposium on Software Reliability Engineering,} \\
        & \textit{ISSRE 2022, Charlotte, NC, USA, October 31 - Nov. 3, 2022.} \\ & \\

\textsc{May} 2022  & \textbf{Fuzzing Class Specifications} \\
	& Facundo Molina, Marcelo d'Amorim and Nazareno Aguirre. \\
	& \textit{Proceedings of the 44th ACM/IEEE International Conference} \\
	& \textit{on Software Engineering, ICSE 2022, Pittsburgh, USA, May 22-27, 2022.} \\ & \\

\textsc{May} 2021  & \textbf{EvoSpex: An Evolutionary Algorithm for Learning Postconditions} \\
& Facundo Molina, Pablo Ponzio, Nazareno Aguirre and Marcelo Frias. \\
& \textit{Proceedings of the 43rd ACM/IEEE International Conference} \\
& \textit{on Software Engineering, ICSE 2021, Madrid, Spain, May 23-29, 2021.} \\ & \\

\textsc{September} 2020  & \textbf{Applying Learning Techniques to Oracle Synthesis} \\
& Facundo Molina \\
& \textit{Doctoral Symposium, Proceedings of the 35th IEEE/ACM International Conference} \\
& \textit{on Automated Software Engineering, ASE 2020, Australia, September 21-25, 2020.} \\ & \\

\textsc{July} 2019  & \textbf{An Evolutionary Approach to Translating Operational Specifications} \\ & \textbf{into Declarative Specifications} - Facundo Molina, César Cornejo, Renzo\\ 
& Degiovanni, Germán Regis, Pablo Castro, Nazareno Aguirre and Marcelo Frias \\
& \textit{Science of Computer Programming, Volume 181, Pages 47-63, 2019.} \\ & \\

\textsc{May} 2019  & \textbf{Training Binary Classifiers as Data Structure Invariants} \\ 
& Facundo Molina, Pablo Ponzio, Renzo Degiovanni, Germán Regis, \\ 
& Nazareno Aguirre and Marcelo Frias \\
& \textit{Proceedings of the 41th International Conference on Software Engineering,} \\
& \textit{ICSE 2019, Montreal, Canada, May 25-31, 2019.} \\ & \\

\textsc{September} 2018  & \textbf{A Genetic Algorithm for Goal-Conflict Identification} \\ 
& Renzo Degiovanni, Facundo Molina, Germán Regis and Nazareno Aguirre \\
& \textit{Proceedings of the 33rd ACM/IEEE International Conference on Automated } \\
& \textit{Software Engineering, ASE 2018, Montpellier, France, September 3-7, 2018.} \\ & \\

\textsc{May} 2018  & \textbf{From Operational to Declarative Specifications using a Genetic} \\ & \textbf{Algorithm} - Facundo Molina, Renzo Degiovanni, Germán Regis, Pablo Castro,\\
& Nazareno Aguirre and Marcelo Frias \\
& \textit{Proceedings of the 11th International Workshop on Search-Based Software Testing,} \\
& \textit{SBST@ICSE 2018, Gothenburg, Sweden, May 28-29, 2018.} \\ & \\

\textsc{November} 2016 & \textbf{An Evolutionary Approach to Translate Operational Specifications} \\ & \textbf{into Declarative Specifications} - Facundo Molina, César Cornejo, Renzo \\
& Degiovanni, Germán Regis, Pablo Castro, Nazareno Aguirre and Marcelo Frias \\
& \textit{Proceedings of the 19th Brazilian Symposium on Formal Methods} \\ 
& \textit{SBMF 2016, Natal, Brazil, November 22-25, 2016.} \\ & \\

\end{longtable}

\section{Public Talks}

\begin{longtable}{rl}
\textsc{October} 2023  & \textbf{Automated Generation of Test Oracles} \\ 
& Invited speaker at the \textit{Jornadas de Ciencias de la Computación}, JCC 2023, \\ 
& Rosario, Argentina. \\ & \\

\textsc{September} 2023  & \textbf{SpecFuzzer: A Tool for Inferring Class Specifications via} \\ 
& \textbf{Grammar-based Fuzzing} - Tool Demonstrations track, ASE conference, Luxembourg. \\ & \\

\textsc{July} 2023 & \textbf{EvoSpex: A Search-based Tool for Postcondition Inference} \\ 
& Tool Demonstrations track, ISSTA conference, Seattle, USA. \\ & \\

\textsc{November} 2022 & \textbf{Learning to Prune Infeasible Paths in Generalized Symbolic Execution} \\
& Research track, ISSRE conference, Charlotte, USA. \\ & \\

\textsc{October} 2022 & \textbf{Fuzzing Class Specifications} - Oral communication, \\ 
& Simposio Argentino de Ingeniería de Software ASSE 2022 (virtual), Argentina. \\ & \\

\textsc{May} 2022 & \textbf{Fuzzing Class Specifications} - Research track, ICSE conference, Pittsburgh, USA. \\ & \\

\textsc{March} 2022 & \textbf{EvoSpex: An Evolutionary Algorithm for Learning Postconditions} \\ 
& Invited talk, Argentine Workshop on Fundamentals for the Automatic Analysis \\ 
& and Construction of Software FACAS 2022, La Falda, Argentina. \\ & \\

\textsc{October} 2021 & \textbf{EvoSpex: An Evolutionary Algorithm for Learning Postconditions} \\ 
& Oral communication, Simposio Argentino de Ingeniería de Software ASSE 2021 (virtual), \\ 
& Argentina. \\ & \\

\textsc{May} 2021 & \textbf{EvoSpex: An Evolutionary Algorithm for Learning Postconditions} \\ 
& Research track, ICSE conference (virtual), Madrid, Spain. \\ & \\

\textsc{September} 2020 & \textbf{Applying Learning Techniques to Oracle Synthesis} \\ 
& Doctoral symposium, ASE conference (virtual), Melbourne, Australia. \\ & \\

\textsc{May} 2019 & \textbf{Training Binary Classifiers as Data Structure Invariants} \\ 
& Research track, ICSE conference, Montréal, Canada. \\ & \\

\textsc{March} 2019 & \textbf{Learning Hybrid Invariants to Improve Symbolic Execution on Structurally} \\ 
& \textbf{Complex Inputs} - Invited talk, Argentine Workshop on Fundamentals for the Automatic \\ 
& Analysis and Construction of Software FACAS 2021, La Falda, Argentina. \\ & \\

\textsc{September} 2018 & \textbf{A Genetic Algorithm for Goal-Conflict Identification} \\ 
& Research track, ASE conference, Montpellier, France. \\ & \\

\textsc{November} 2016 & \textbf{An Evolutionary Approach to Translate Operational Specifications into} \\
& \textbf{Declarative Specifications} - Research track, Brazilian Symposium on Formal \\ 
& Methods SBMF 2016, Natal, Brazil. \\ & \\

\end{longtable}

\section{Research Prototypes}
\begin{longtable}{rl}

%\textbf{FixCheck} & FixCheck is a tool \\ & \\

\textbf{SpecFuzzer} & SpecFuzzer is a tool that automatically infers test oracles in the form of class specifica-\\ 
& tions (postconditions, invariants), and works for Java classes. SpecFuzzer uses a fuzzer \\
& as a generator of candidate assertions derived from a grammar that is automatically \\
& obtained from the class definition; a dynamic invariant detector –Daikon– to filter out \\
& assertions invalidated by a test suite; and a mutation-based mechanism to cluster and \\
& rank assertions, so that similar constraints are grouped and then the stronger prioritized. \\ 
& SpecFuzzer is available at: \href{https://github.com/facumolina/specfuzzer}{https://github.com/facumolina/specfuzzer} \\ & \\


\textbf{EvoSpex} & EvoSpex is a tool that, given a Java method, uses an evolutionary algorithm to produce \\ 
& a specification of the method's current behavior, in the form of postcondition assertions. \\
& EvoSpex is available at: \href{https://github.com/facumolina/evospex}{https://github.com/facumolina/evospex} \\ & \\

\textbf{PLI} & PLI is an efficient symbolic execution approach for programs that manipulate complex \\
& heap-allocated data structures with rich structural constraints. PLI works for Java, \\
& and allows preconditions to be specified as standard operational predicates for concrete \\
& structures, eliminating the need for additional specifications tailored to symbolic heaps. \\
& PLI is available at: \href{https://github.com/JuanmaCopia/spf-pli}{https://github.com/JuanmaCopia/spf-pli} \\ & \\

\end{longtable}

\section{Teaching Background}
\begin{longtable}{rl}

\textsc{March 2018} & \textbf{Teaching Assistant} \\
\textsc{- June 2022} & \textsc{Courses:} Computability and Complexity, Distributed and Outsourced \\
& Software Engineering, Introduction to Programming. \\
& \textit{Department of Computer Science - FCEFQyN} \\
& \textit{University of Río Cuarto} - Argentina \\ & \\

\textsc{August 2014} & \textbf{Student Teaching Assistant} \\
\textsc{- June 2017} & \textsc{Courses:} Data Structures and Algorithms, Algorithms Design Techniques, \\
& Programming Paradigms, and System Design and Analysis. \\
& \textit{Department of Computer Science - FCEFQyN} \\
& \textit{University of Río Cuarto} - Argentina \\ & \\

\end{longtable}

\section{Supervised Students}
\begin{longtable}{rl}

\textsc{2023 - Present} & \textbf{Agustin Nolasco} - \textit{Bachelor student} - University of Río Cuarto, Argentina. \\ 
& Agustin's thesis presents a new technique for the inference of metamorphic oracles, \\ 
& based on runtime analysis, grammar-based fuzzing and SAT solving. \\ & \\

\textsc{2023 - Present} & \textbf{Claudio Dosantos} - \textit{Bachelor student} - University of Río Cuarto, Argentina. \\
& Claudio's thesis aims to analyze the effectiveness of regression testing when using \\ 
& different kind of oracles, such as unit assertions and contracts. \\ & \\

\textsc{2023 - Present} & \textbf{Ignacio Gonzalez} - \textit{Bachelor student} - University of Río Cuarto, Argentina. \\
& Automated test generation tools play a crucial role on dynamic specification inference \\ 
& techniques. Ignacio's work aims at studying how different test generation approaches \\ 
& affects the effectiveness of specification inference techniques. \\ & \\

\end{longtable}

\section{Academic Service}
\begin{longtable}{rl}

\textsc{Since 2021}   & \textbf{Program Committee} \\
& \textit{International Conference on AI Foundation Models and Software Engineering, FORGE 2024.} \\
& \textit{International Working Conference on Source Code Analysis and Manipulation, SCAM 2023.} \\ 
& \textit{International Workshop on Test Oracles, TORACLE 2021 (at ESEC/FSE 2021).} \\ & \\

\textsc{Since 2023} & \textbf{Artifact Evaluation Committee} \\
& \textit{International Conference on Software Engineering, ICSE 2024.} \\
& \textit{International Symposium on Software Testing and Analysis, ISSTA 2023.} \\ 
& \textit{Static Analysis Symposium, SAS 2023.} \\ & \\

\textsc{2017 - 2022} & \textbf{Student Volunteer} \\
& \textit{International Conference on Software Engineering, ICSE 2022.} \\
& \textit{International Conference on Software Engineering, ICSE 2021.} \\
& \textit{International Conference on Software Engineering, ICSE 2019.} \\
& \textit{International Conference on Automated Software Engineering, ASE 2018.} \\
& \textit{International Conference on Software Engineering, ICSE 2017.} \\ & \\

\end{longtable}

\section{Awards \& Distinctions}
\begin{longtable}{rl}
\\
\textsc{2020} & \textbf{Latin America PhD Award} \\
& \textit{A research award for PhD students in computing related fields in their 3rd year} \\
& \textit{or beyond at universities in Latin America, and granted by Microsoft Research.} \\ & \\

\textsc{2018} & \textbf{Best Paper Award} \\ 
& \textit{From Operational to Declarative Specifications using a Genetic Algorithm} \\
& \textit{11th International Workshop on Search-Based Software Testing, SBST 2018.} \\ & \\

\textsc{2016} & \textbf{Best Paper Award} \\ 
& \textit{An Evolutionary Approach to Translate Operational Specifications into } \\
& \textit{Declarative Specifications, 19th Brazilian Symposium on Formal Methods, SBMF 2016.} \\ & \\

\textsc{2016} & \textbf{University of Rio Cuarto flag bearer for a 1-year period} \\ 
& \textit{Traditional honour in educational institutions in Argentina to the three top students} \\ 
& \textit{in the institution.} \\ & \\
\end{longtable}

\section{Grants \& Scholarships}
\begin{tabular}{rl}
\\
\textsc{2017} & \textbf{Doctoral Scholarship} \\ 
& \textit{5-year Scholarship granted by Argentina's National Scientific and Technical Research} \\
& \textit{Council (CONICET) to fund doctoral students.} \\ & \\

\textsc{2016} & \textbf{EVC-CIN Scholarship} \\ 
& \textit{1-year Scolarship granted by the argentinian National Inter University Council (CIN)} \\
& \textit{to encourage undergraduate students to pursue scientific vocations.} \\ & \\

\end{tabular}

\section{Extracurricular Courses Taken}
\begin{longtable}{rl}
\textsc{October} 2019 & \textbf{Neural Networks and Deep Learning} - Adjunct Professor Andrew Ng \\
& \textit{Foundations of Deep Learning}  \\
& \textit{An online non-credit course authorized by deeplearning.ai} \\
& \textit{Coursera} \\ & \\

\textsc{March} 2019 & \textbf{Introduction to Data Science in Python} - Christopher Brooks \\
& \textit{Introduction to data manipulation and cleaning techniques using pandas}  \\
& \textit{An online non-credit course authorized by University of Michigan} \\
& \textit{Coursera} \\ & \\

\textsc{August} 2018 & \textbf{Neural Networks} - Dr. Francisco Tamarit \\
\textsc{November} 2018 & \textit{Mathematical Foundations of Artificial Neural Networks}  \\ 
& \textit{Postgraduate courses} \\
& \textit{University of Córdoba} - Argentina \\ & \\

\textsc{August} 2017 & \textbf{Text Mining} - Dr. Laura Alonso Alemany \\
\textsc{November} 2017 & \textit{Text Mining techniques applied to Natural Language Processing problems (Word}  \\ 
& \textit{similarity, Document clustering, Sense discrimination, Machine translation)} \\
& \textit{Postgraduate courses} \\
& \textit{University of Córdoba} - Argentina \\ & \\

\textsc{August} 2017 & \textbf{Information and its Demons} - Dr. Javier Blanco \\
\textsc{November} 2017 & \textit{Information Philosophy} \\
& \textit{Postgraduate courses} \\
& \textit{University of Río Cuarto} - Argentina \\ & \\

\textsc{February} 2017 & \textbf{Human Dynamics: Data, Networks and Modelling} - Dr. Márton Karsai \\
& \textit{Summer School of Computer Science RIO 2017} \\
& \textit{University of Río Cuarto} - Argentina \\ & \\

\textsc{March} 2016 & \textbf{Software Testing} - Dr. Renzo Degiovanni \\
\textsc{June} 2016 & \textit{Main software testing techniques using state-of-the-art tools} \\
& \textit{Postgraduate courses} \\
& \textit{University of Río Cuarto} - Argentina \\ & \\

\textsc{February} 2016 & \textbf{Systematic Test Case Generation} - Prof. Sarfraz Khurshid \\
& \textit{Summer School of Computer Science RIO 2016} \\
& \textit{University of Río Cuarto} - Argentina \\ & \\

\textsc{February} 2016 & \textbf{Symbolic Program Analysis} - Prof. Willem Visser \\
& \textit{Summer School of Computer Science RIO 2016} \\
& \textit{University of Río Cuarto} - Argentina \\ & \\

\textsc{February} 2015 & \textbf{Description Logic Reasoning} - Dr. Anni-Yasmin Turhan\\
& \textit{Summer School of Computer Science RIO 2015} \\
& \textit{University of Río Cuarto} - Argentina \\ & \\

\textsc{February} 2015 & \textbf{Fundamentals of Quantum Programming Languages} - Dr. Alejandro Díaz-Caro \\
& \textit{Summer School of Computer Science RIO 2015} \\
& \textit{University of Río Cuarto} - Argentina \\ & \\

\end{longtable}

\section{Languages}
\begin{tabular}{rl}
\\
\textsc{Spanish:} & Mother tongue\\
\textsc{English:} & Fluent \\ & \\
\end{tabular}

\end{document}
