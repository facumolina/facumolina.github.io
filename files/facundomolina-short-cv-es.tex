%%%%%%%%%%%%%%%%%%%%%%%%%%%%%%%%%%%%%%%%%
% Plasmati Graduate CV
% LaTeX Template
% Version 1.0 (24/3/13)
%
% This template has been downloaded from:
% http://www.LaTeXTemplates.com
%
% Original author:
% Alessandro Plasmati (alessandro.plasmati@gmail.com)
%
% License:
% CC BY-NC-SA 3.0 (http://creativecommons.org/licenses/by-nc-sa/3.0/)
%
% Important note:
% This template needs to be compiled with XeLaTeX.
% The main document font is called Fontin and can be downloaded for free
% from here: http://www.exljbris.com/fontin.html
%
%%%%%%%%%%%%%%%%%%%%%%%%%%%%%%%%%%%%%%%%%

%----------------------------------------------------------------------------------------
%	PACKAGES AND OTHER DOCUMENT CONFIGURATIONS
%----------------------------------------------------------------------------------------

\documentclass[a4paper,10pt]{article} % Default font size and paper size

\usepackage[utf8]{inputenc}

\usepackage{amsfonts}
\usepackage{longtable}
\usepackage[usenames,dvipsnames]{xcolor} % Required for specifying custom colors

\usepackage{fullpage}
% To reduce the height of the top margin uncomment: \addtolength{\voffset}{-1.3cm}

\usepackage{hyperref} % Required for adding links	and customizing them
\definecolor{linkcolour}{rgb}{0,0.2,0.6} % Link color
\hypersetup{colorlinks,breaklinks,urlcolor=linkcolour,linkcolor=linkcolour} % Set link colors throughout the document

\usepackage{titlesec} % Used to customize the \section command
\titleformat{\section}{\Large\scshape\raggedright}{}{0em}{}[\titlerule] % Text formatting of sections
\titleformat{\subsection}{\large\scshape\raggedright}{}{1em}{\underline}[] % Text formatting of subsections
\titlespacing{\section}{0pt}{3pt}{5pt} % Spacing around sections
\titlespacing{\subsection}{0pt}{10pt}{2pt} % Spacing around subsections
\usepackage{graphicx}
\begin{document}

\pagestyle{empty} % Removes page numbering

%----------------------------------------------------------------------------------------
%	NAME AND CONTACT INFORMATION
%----------------------------------------------------------------------------------------

{\raggedleft{\Huge Facundo \textsc{Molina}}} \\
\textit{Investigador Postdoctoral} \\
\textsc{Email:} \href{mailto:facundo.molina@imdea.org}{facundo.molina@imdea.org} \\
\textsc{Website:} \href{https://facumolina.github.io}{https://facumolina.github.io}

\section{Experiencia Laboral}

\begin{tabular}{rl}
\\
\textsc{Desde 2022}	& \textbf{Investigador Postdoctoral} \\
& \textit{IMDEA Software Institute}, Madrid, España. \\

\textsc{2018 - 2022} & \textbf{Asistente Docente} \\ 
& \textit{Departamento de Computación, Universidad Nacional de Río Cuarto}, Argentina. \\

\textsc{2014 - 2017} & \textbf{Asistente Docente (estudiante)} \\
& \textit{Departamento de Computación, Universidad Nacional de Río Cuarto}, Argentina. \\

\end{tabular}

%----------------------------------------------------------------------------------------
%	EDUCATION
%----------------------------------------------------------------------------------------

\section{Educación}

\begin{tabular}{rl}
\\
\textsc{2017 - 2022}	& \textbf{Doctor en Ciencias de la Computación} \\
& \textit{Facultad de Matemática, Astronomía, Física y Computación, Universidad Nacional} \\ 
& \textit{de Córdoba}, Argentina. \\

\textsc{2012 - 2017}	& \textbf{Licenciado en Ciencias de la Computación} (5 años + tesis) \\
& \textit{Departamento de Ciencias de la Computación, Universidad Nacional de Río Cuarto}, \\
& Argentina. Promedio: 9.43. \\

\textsc{2012 - 2014}	& \textbf{Analista en Computación} (3 años + proyecto final) \\
& \textit{Departamento de Computación, Universidad Nacional de Río Cuarto}, \\ 
& Argentina. Promedio: 9.33. \\

\end{tabular}

\section{Publicaciones}
\begin{longtable}{rl}

\textsc{Noviembre} 2022  & \textbf{Learning to Prune Infeasible Paths in Generalized Symbolic Execution} \\
& F. Molina, P. Ponzio, N. Aguirre and M. Frias. \textit{{IEEE} 33rd International Symposium} \\
& \textit{on Software Reliability, {ISSRE} 2022, Charlotte, USA.} \\

\textsc{Mayo} 2022  & \textbf{Fuzzing Class Specifications}. F. Molina, M. d'Amorim and N. Aguirre. \\
& \textit{ACM/IEEE 44th International Conference on Software Engineering, ICSE 2022,} \\
& \textit{Pittsburgh, USA.} \\

\textsc{Mayo} 2021  & \textbf{EvoSpex: An Evolutionary Algorithm for Learning Postconditions} \\
& F. Molina, P. Ponzio, N. Aguirre and M. Frias. \textit{ACM/IEEE International Conference} \\
& \textit{on Software Engineering, ICSE 2021, Madrid, Spain.} \\

\textsc{Septiembre} 2020  & \textbf{Applying Learning Techniques to Oracle Synthesis}. F. Molina. \\
& \textit{Doctoral Symposium, {IEEE/ACM} 35th International Conference on Automated} \\
& \textit{Software Engineering, ASE 2020, Australia.} \\

\textsc{Julio} 2019  & \textbf{An Evolutionary Approach to Translating Operational Specifications} \\ 
& \textbf{into Declarative Specifications}. F. Molina, C. Cornejo, R. Degiovanni, \\ 
& G. Regis, P. Castro, N. Aguirre and M. Frias. \textit{Science of Computer Programming 2019.} \\

\textsc{Mayo} 2019  & \textbf{Training Binary Classifiers as Data Structure Invariants} \\ 
& F. Molina, P. Ponzio, R. Degiovanni, G. Regis, N. Aguirre and M. Frias \\ 
& \textit{International Conference on Software Engineering, ICSE 2019, Montreal, Canada.} \\

\textsc{Septiembre} 2018  & \textbf{A Genetic Algorithm for Goal-Conflict Identification} \\
& R. Degiovanni, F. Molina, G. Regis and N. Aguirre. \textit{{ACM/IEEE} 33rd International} \\
& \textit{Conference on Automated Software Engineering, ASE 2018, Montpellier, France.} \\

\textsc{Mayo} 2018 & \textbf{From Operational to Declarative Specifications using a Genetic Algorithm} \\ 
& F. Molina, R. Degiovanni, G. Regis, P. Castro, N. Aguirre and M. Frias. \textit{11th International}\\
& \textit{Workshop on Search-Based Software Testing, SBST@ICSE 2018, Gothenburg, Sweden.} \\

\textsc{Noviembre} 2016 & \textbf{An Evolutionary Approach to Translate Operational Specifications into} \\ 
& \textbf{Declarative Specifications}. F. Molina, C. Cornejo, R. Degiovanni, G. Regis, \\ 
& P. Castro, N. Aguirre and M. Frias. \textit{19th Brazilian Symposium on Formal Methods,} \\ 
& \textit{SBMF 2016, Natal, Brazil.} \\

\end{longtable}

%\section{Recent Funding}
%\begin{longtable}{rl}

%\textsc{2022} & \textbf{Modular Bounded Verification with Expressive Contracts} \\
%& \textit{Amazon Research Awards ARA 2022. Amazon program providing unrestricted funds} \\ 
%& \textit{and AWS Promotional Credits to academic researchers.} \\
%& Principal Investigator: Marcelo Frias. \\

%\end{longtable}

\section{Software}

\begin{longtable}{rl}
\textsc{ISSRE 2022} & \textbf{\url{https://sites.google.com/view/learning-symbolic-invariants}} \\
& \textit{Técnica de aprendizaje automático para aprender a podar caminos inviables} \\
& \textit{en ejecución simbólica generalizada, presenteada en ISSRE 2022.} \\

\textsc{ICSE 2022} & \textbf{SpecFuzzer: \url{https://sites.google.com/view/specfuzzer}} \\
& \textit{Técnica basada en fuzzing para inferir especificaciones de clase, presentada en ICSE 2022.} \\

\textsc{ICSE 2021} & \textbf{EvoSpex: \url{https://sites.google.com/view/evospex}} \\
& \textit{Algoritmo evolutivo para aprender post-condiciones de métodos Java, presentado en ICSE 2021.} \\

\textsc{ICSE 2019} & \textbf{\url{https://sites.google.com/site/learninginvariants}} \\
& \textit{Técnica para entrenar modelos de aprendizaje automático para capturar invariantes de} \\ 
& \textit{estructuras de datos, presentada en ICSE 2019.} \\

\end{longtable}

\section{Servicios}

\begin{longtable}{rl}
\textsc{2021}   & \textbf{Comité de Programa} \\
& \textit{1st International Workshop on Test Oracles, TORACLE 2021 (at ESEC/FSE 2021).} \\

\textsc{2017 - 2022} & \textbf{Estudiante Voluntario} \\
& \textit{International Conference on Software Engineering - ICSE 2017, ICSE 2019,} \\
& \textit{ICSE 2021, ICSE 2022.} \\
& \textit{International Conference on Automated Software Engineering - ASE 2018.} \\

\end{longtable}

\section{Premios \& Distinciones}
\begin{longtable}{rl}

\textsc{2020} & \textbf{Latin America PhD Award} \\
& \textit{Premio a estudiantes de doctorado de universidades de América Latina, otorgado} \\
& \textit{por Microsoft Research.} \\

\textsc{2018} & \textbf{Best Paper Award} \\
& \textit{From Operational to Declarative Specifications using a Genetic Algorithm, 11th} \\
& \textit{International Workshop on Search-Based Software Testing, SBST 2018.} \\

\textsc{2016} & \textbf{Best Paper Award} \\
& \textit{An Evolutionary Approach to Translate Operational Specifications into } \\
& \textit{Declarative Specifications, 19th Brazilian Symposium on Formal Methods, SBMF 2016.} \\

\textsc{2016} & \textbf{Escolta de la bandera de la Universidad Nacional de Río Cuarto} \\
& \textit{Honor tradicional en instituciones educativas de Argentina otorgado a los tres estudiantes} \\
& \textit{más sobresalientes de la institución.} \\

\end{longtable}

\section{Becas \& Subvenciones}
\begin{longtable}{rl}

\textsc{2017} & \textbf{Beca Doctoral} \\
& \textit{Beca de 5 años otorgada por el Consejo Nacional de Investigaciones Científicas y Técnicas} \\
& \textit{(CONICET) de Argentina, para financiar a estudiantes de doctorado.} \\

\textsc{2016} & \textbf{Beca EVC-CIN} \\
& \textit{Beca de 1 año otorgada por el Consejo Interuniversitario Nacional (CIN) de Argentina, } \\
& \textit{para animar a los estudiantes de grado a seguir vocaciones científicas.} \\

\end{longtable}

\section{Idiomas}

\begin{tabular}{rl}

\textsc{Español:} & lengua madre \\
\textsc{Inglés:} & fluído \\

\end{tabular}

\end{document}
